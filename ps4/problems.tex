\section{Problems}

Here is a link to my code: \url{https://github.com/filipmellgren/QMM/tree/main/ps4}.

\begin{questions}
\question{Life cycle questions}
\begin{solution}
\includegraphics[scale=0.5]{figures/mean_cons.png}

\includegraphics[scale=0.5]{figures/mean_income.png}

\includegraphics[scale=0.5]{figures/mean_assets.png}

\includegraphics[scale=0.5]{figures/fraction_alive.png}

I don't understand what is meant in 1.3.2, so I skip that.

\includegraphics[scale=0.5]{figures/hh_heterogeneity.png}

In the initial periods, there is more consumption heterogeneity between households than differences in income. Over time, households diverge but their consumption diverge less than income, thanks to their assets which help them insure against shocks. 

\end{solution}

\question{Insurance questions}
\begin{solution}
A higher rho (i.e. persistence in income shocks), tends to be associated with a shrinking ratio of $\frac{\operatorname{cov}\left(\Delta c_{i t}, n_{i t}\right)}{\operatorname{var}\left(n_{i t}\right)}$, which is a measure of insurance. If the correlation had been zero, agents wouldn't change their consumption following a shock, so insurance is perfect. 

I use the same shocks for all parameterisations and the same random seed to draw assets. Not sure why relationship is not monotone. 

\includegraphics[scale=0.5]{figures/insurance_question.png}

\end{solution}
\end{questions}


