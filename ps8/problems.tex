\section{Firms and inventories}

Table for $\bar{\xi} = 0.2198$. Associated price is 3.2297.

\begin{table}[!ht]
    \centering
    \begin{tabular}{|l|l|l|l|l|l|}
    \hline
        ~ & 0 & 1 & 2 & 3 & 4 \\ \hline
        s & 1.19132 & 0.72175 & 0.33405 & 0.09706 & 0.00818 \\ \hline
        Mass & 0.34226 & 0.31462 & 0.2134 & 0.09039 & 0.03933 \\ \hline
        m & 0.46957 & 0.38771 & 0.23698 & 0.08888 & 0.01997 \\ \hline
    \end{tabular}
\end{table}

Table for $\bar{\xi} = 0.333$. Associated price is 3.2996.

\begin{table}[!ht]
    \centering
    \begin{tabular}{|l|l|l|l|l|l|}
    \hline
        ~ & 0 & 1 & 2 & 3 & 4 \\ \hline
        s & 1.4222 & 0.94544 & 0.52923 & 0.223 & 0.05936 \\ \hline
        Mass & 0.26914 & 0.25867 & 0.213 & 0.13387 & 0.12533 \\ \hline
        m & 0.47676 & 0.41621 & 0.30623 & 0.16364 & 0.05891 \\ \hline
    \end{tabular}
\end{table}

Intuitively, a higher bar for what cost shocks are possible, shifts the distribution of cost shocks upward making it more costly for firms to adjust, on average. They should therefore adjust less frequently. Now, if we let shocks be correlated across firms, we'd expect longer periods of relatively little adjustment, and some periods where many firms choose to adjust, thereby exacerbating the business cycles.

We also note that firms tend to reset to a higher inventory with higher cost draws, which should increase volatility of the business cycle. 




